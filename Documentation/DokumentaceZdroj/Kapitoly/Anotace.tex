\thispagestyle{empty}
\setcounter{page}{0}

\section*{Anotace}
Tato práce se zabývá vývojem a implementací hry Snake, a umělé inteligence ovládající hada tak, že zaplní celé hrací pole, aniž by došlo ke kolizi. Hlavním cílem je nalezení optimální cesty k jablku a minimalizace rizika srážky prostřednictvím různých algoritmů. V rámci práce budou analyzovány a porovnány různé metody plánování pohybu, jako je A* algoritmus a algoritmy pro hledání Hamiltonovské kružnice. Výsledkem práce bude nejen funkční implementace hry, ale také vyhodnocení efektivity jednotlivých přístupů.

\section*{Abstract}
This work focuses on developing and implementing a Snake game in which the snake plays autonomously and completes the game by filling the entire playing field without colliding. The main objective is to find an optimal path to the apple while minimizing the risk of collision using various algorithms. The work analyzes and compares different pathfinding methods, such as the A* algorithm and algorithms for finding Hamiltonian cycles. The result of this work is a functional implementation of the game together with an evaluation of the efficiency of different approaches.

\section*{Anmerkung}
Diese Arbeit befasst sich mit der Entwicklung und Implementierung eines Snake-Spiels, in dem die Schlange autonom spielt und das Spiel beendet, indem sie das gesamte Spielfeld füllt, ohne mit sich selbst zu kollidieren. Das Hauptziel ist es, einen optimalen Weg zum Apfel zu finden und gleichzeitig das Kollisionsrisiko durch verschiedene Algorithmen zu minimieren. In dieser Arbeit werden verschiedene Methoden zur Pfadsuche analysiert und verglichen, darunter der A*-Algorithmus und Algorithmen zur Suche nach Hamiltonschen Zyklen. Das Ergebnis dieser Arbeit wird nicht nur eine funktionale Implementierung des Spiels sein, sondern auch eine Bewertung der Effizienz verschiedener Ansätze.